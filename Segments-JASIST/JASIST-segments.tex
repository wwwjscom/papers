\documentclass{endm}
\usepackage{endmmacro}
\usepackage{graphicx}
\usepackage{indentfirst}
\usepackage{lscape}
\usepackage{CJK}

\usepackage{amsmath}
\usepackage{subfigure}
\usepackage{hyperref}
\usepackage{multirow}
% The following is enclosed to allow easy detection of differences in
% ascii coding.
% Upper-case    A B C D E F G H I J K L M N O P Q R S T U V W X Y Z
% Lower-case    a b c d e f g h i j k l m n o p q r s t u v w x y z
% Digits        0 1 2 3 4 5 6 7 8 9
% Exclamation   !           Double quote "          Hash (number) #
% Dollar        $           Percent      %          Ampersand     &
% Acute accent  '           Left paren   (          Right paren   )
% Asterisk      *           Plus         +          Comma         ,
% Minus         -           Point        .          Solidus       /
% Colon         :           Semicolon  7  ;          Less than     <
% Equals        =           Greater than >          Question mark ?
% At            @           Left bracket [          Backslash     \
% Right bracket ]           Circumflex   ^          Underscore    _
% Grave accent  `           Left brace   {          Vertical bar  |
% Right brace   }           Tilde        ~

\newcommand{\Nat}{{\mathbb N}}
\newcommand{\Real}{{\mathbb R}}
\def\lastname{Ye}

\begin{document}

\begin{CJK}{UTF8}{song}
% DO NOT REMOVE: Creates space for Elsevier logo, ScienceDirect logo
% and ENDM logo
\begin{verbatim}\end{verbatim}\vspace{2.5cm}


\begin{frontmatter}
% \title{A Graph based Approach to Mine Multilingual Word Associations from Wikipedia}
\title{Mining Cross-language Association Dictionary from Wikipedia for Cross-Language Information Retrieval}


\author{Zheng Ye \thanksref{myemail}}
\address{School of Information Technology\\ York University\\ Toronto, Ontario, M3J 1P3, Canada}


\author{Jimmy Huang \thanksref{coemail}}
\address{School of Information Technology\\ York University\\
   Toronto, Ontario, M3J 1P3, Canada} 

\author{Hongfei Lin \thanksref{coemail1}}
\address{Department of Computer Science and Engineering\\ Dalian University of Technology\\ Dalian, Liaoning, 116023, China}

\author{Ben He\thanksref{coemail2}}
\address{School of Information Technology\\ York University\\
   Toronto, Ontario, M3J 1P3, Canada} 
\thanks[myemail]{Email:
   \href{mailto:yezheng@yorku.ca} {\texttt{\normalshape
   yezheng@yorku.ca}}} 
\thanks[coemail]{Email:
   \href{mailto:jhuang@yorku.ca} {\texttt{\normalshape
   jhuang@yorku.ca}}}
\thanks[coemail1]{Email:
   \href{mailto:hflin@dlut.edu.cn} {\texttt{\normalshape
   hflin@dlut.edu.cn}}}
\thanks[coemail2]{Email:
   \href{mailto:benhe@yorku.ca} {\texttt{\normalshape
   benhe@yorku.ca}}}


\newpage
\begin{abstract}
Wikipedia has various impressive characteristics, such as a dense link structure and a huge amount of articles in
different languages, which make it a notable Web corpus for knowledge extraction and mining.
In this paper, we focus on exploring its potential of mining a cross-language association dictionary (CLAD). 
In particular, based on our analysis of Wikipedia structure and motivated by a
psychological theory of word meaning, we propose a graph-based approach to constructing a
cross-language association dictionary from Wikipedia, which could be used in a series of cross language accessing or
processing applications. In order to evaluation the quality of the mined CLAD, we explore two different applications 
in the cross-language information retrieval (CLIR). First, we use the obtained CLAD to conduct cross-language
query expansion experiments; and second, we use it to  filter out translation candidates with low translation
probabilities. Experimental results on a variety of standard CLIR test collections show that the CLIR retrieval
performance can be significantly improved on both two applications of CLAD. We conclude that it is possible to mine a
cross-language association dictionary with good quality from Wikipedia. 

\end{abstract}
\begin{keyword}
Association Dictionary, Wikipedia, CLIR.
\end{keyword}

\end{frontmatter}

\section{Introduction}\label{s:introduce}


\bibliographystyle{endm}
\bibliography{ref}


\end{CJK}

\end{document}